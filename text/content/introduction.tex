%!TEX root = ../thesis.tex
\chapter{Introduction}
\label{ch:introduction}

Modern interactive web applications have a growing frontend code base and become very complex. This makes them harder to maintain and to find suitable structures. In the past few years, several frameworks have been introduced to deal with the growing code base in a structured way. Many of these frameworks are based on structures derived from backend frameworks and their MVC structure (see \refchapter{sec:mvcmvvm}). BackboneJS\footnote{\url{https://backbonejs.org}, last-checked on 21/04/2014} for example is one of the most commonly used frameworks \cite{sanderson2012rich}. It structures applications into models, routers and views. Therefore it is similar to classic web frameworks such as Ruby on Rails\footnote{A ruby-based web framework, \url{http://rubyonrails.org/}, last checked on 24/04/2014} or Django\footnote{A python-based web framework, \url{https://www.djangoproject.com/}, last checked on 24/04/2014}. In addition to that, all components in BackboneJS applications are developed object-oriented and they are composed from the above described base components or components that are derived from them. This approach is somehow comparable to developing SWING\footnote{A Java-based GUI toolkit} applications, although it is much more flexible due to BackbonJS's lean structure and JavaScript's prototype model.

This report will focus on another framework with a different approach. It will give an in-depth look into the framework AngularJS\footnote{\url{https://angularjs.org/}, last-checked on 21/04/2014} which is based on the MVVM pattern \cite{minar2012mvvm}. It is also based on concepts like `dependency injection' and `separation of concerns' which are not common in other frameworks but which give developers the opportunity to write easily testable and highly reusable applications and components.

AngularJS's basic concepts and its core principles will be explained in text and in two in-depth practical examples (\refchapter{part:xbmc} and \refchapter{part:collection}). The code for both examples is licensed under an open source license and both chapters link to the published source code and the deployed versions of the applications. The chapters are written in close relation to the source files and it can be helpful to understand the concepts of AngularJS by reading the source code in addition to the provided code examples in this document. \todo{Link to specific chapters and sections}

\section{MVC and MVVM}
\label{sec:mvcmvvm}

\subsection{MVC}
\label{subsec:mvc}

\begin{figure}[htb]
  \centerline{\includegraphics[width=0.6\linewidth]{images/MVC.pdf}}
  \caption[MVC]{MVC}
  \label{fig:mvc}
\end{figure}

In order to understand the MVVM pattern, on which AngularJS is based, the MVC pattern needs to be explained in advance, because MVVM bases on MVC. MVC stands for ``Model View Controller'' and is a software design pattern which is widely used in many different applications. It consists of three different main parts:

\textbf{The Model}, which represents the the application's state. Models take care of persisting, fetching and updating the state and encapsulate all these transactions in an API.

\textbf{The View}, which is the visual representation of the data and an interface to trigger data changes.

\textbf{The Controller}, which represents the connection between \textbf{Model} and \textbf{View}. It takes care of updating view state and interacts with the model.

They way in which these components interact is strictly limited to the actions that are shown in \reffigure{fig:mvc}. The view never directly interacts with the model and vice versa. In this way, the entire model layer can be replaced by a completely different implementation and the view code would not need to be changed because the view only interacts with the controller. In a best case scenario, even the controller code would not require alterations if the same interface between it and the model would be used again. Basically, any of the three components can be replaced by another implementation and the other components would still work.

Although the basic structure of MVC is well-defined, it is discussed whether the business logic should reside in the model layer or if it should reside in the controller layer. Frameworks often decide to have either ``fat models and skinny controllers'' (e.g. in Ruby on Rails) or to give controllers more responsibilities than just fetching and updating models.

\subsection{MVVM}
\label{subsec:mvvm}

\begin{figure}[htb]
  \centerline{\includegraphics[width=0.6\linewidth]{images/MVVM.pdf}}
  \caption[MVVM]{MVVM}
  \label{fig:mvvm}
\end{figure}

The MVVM pattern is an addition to the MVC pattern and stands for ``Model View ViewModel''. It has been first described by \cite{fowler04mvvm} and was called ``Presentation Model'' initially. Its idea is to add a ``ViewModel'' to the view layer which is another, logical, representation of the model in the backend. The ViewModel is capable of communicating to the data provider (in this case the ``Model'' \reffigure{fig:mvvm}) either directly or through a custom controller layer. The View in this concept often interacts with its ViewModel directly through a binding that takes care of updating the view whenever the model changes and that also updates the model whenever the view state changes.

The advantage of this approach is that the ViewModel acts as an automatic ``translator'' of the model data and the view's visible data representation. Additionally, one ViewModel instance could be used by several (different and independent) View instances so that the state of a specific model is always consistent no matter which View is currently editing it.