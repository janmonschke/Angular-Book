%!TEX root = ../thesis.tex
\chapter{Introduction}
\label{ch:introduction}

Modern interactive web applications have a growing frontend code base and become very complex. This makes them harder to maintain and to find suitable structures. In the past few years, several frameworks have been introduced to deal with the growing code base in a structured way. Many of these frameworks are based on structures derived from backend frameworks and their MVC structure (see \refchapter{sec:mvcmvvm}). BackboneJS\footnote{\url{https://backbonejs.org}, last-checked on 21/04/2014} for example is one of the most commonly used frameworks \cite{sanderson2012rich}. It structures applications into models, routers and views. Therefore it is similar to classic web frameworks such as Ruby on Rails\footnote{A ruby-based web framework, \url{http://rubyonrails.org/}, last checked on 24/04/2014} or Django\footnote{A python-based web framework, \url{https://www.djangoproject.com/}, last checked on 24/04/2014}. In addition to that, all components in BackboneJS applications are developed object-oriented and they are composed from the above described base components or components that are derived from them. This approach is somehow comparable to developing SWING\footnote{A Java-based GUI toolkit} applications, although it is much more flexible due to BackbonJS's lean structure and JavaScript's prototype model.

This report will focus on another framework with a different approach. It will give an in-depth look into the framework AngularJS\footnote{\url{https://angularjs.org/}, last-checked on 21/04/2014} which is based on the MVVM pattern. It is also based on concepts like `dependency injection' and `separation of concerns' which are not common in other frameworks but which give developers the opportunity to write easily testable and highly reusable applications and components.

AngularJS's basic concepts, its core principles will be explained in text and in two in-depth practical examples (\refchapter{part:xbmc} and \refchapter{part:collection}). \todo{Link to specific chapters and sections}

\section{MVC and MVVM}
\label{sec:mvcmvvm}

\begin{figure}[htb]
  \centerline{\includegraphics[width=0.6\linewidth]{images/MVC.pdf}}
  \caption[MVC]{MVC}
  \label{fig:polling}
\end{figure}