%!TEX root = ../../thesis.tex
\section{Components}

The implementation of XBMCMagic was an experiment to write as less JavaScript code as possible and to use AngularJS's expressions\todo{link to expressions description} wherever it was possible. Most frontend code has been achieved by solely using expressions, built-in directives and lean controllers which mostly act as glue code to services or third party libraries.

\subsection{Controllers}

\begin{figure}[htb]
  \centerline{
    \includegraphics[width=0.4\linewidth]{images/xbmc-magic-screen-1.png}
    \includegraphics[width=0.41\linewidth]{images/xbmc-magic-screen-2.png}
  }
  \caption[left: initial screen, right: settings screen]{left: initial screen, right: settings screen}
  \label{fig:xbmcmagic-controllers}
\end{figure}

The app consists of only four controllers which are shown in \reffigure{fig:xbmcmagic-controllers}:

\subsubsection{TabController}

The TabController takes care of switching the currently visible tab and consists only of expressions in the template and one line of JavaScript in the controller file itself (\code{\$scope.currentTab = 0}). The controller is registered in the first line of \reflisting{lst:tab-controller} and from then on the logic relies entirely on directives and expressions. Line 9 switches (\code{ng-switch}) the shown content tab based on the value of \code{currentTab}, which is changed on click (\code{ng-click}) on one of the tab headers in line 3 or in line 6. Furthermore, the visual state of a tab header is changed according to the state of \code{currentTab} as well by using the \code{ng-class} directive.

\begin{lstlisting}[language=HTML, caption=TabController expressions, label=lst:tab-controller]
  <div class="tabs" ng-controller="TabController">
    <div class="tab-header">
        <div  ng-class="{`active': currentTab == 0}"
              ng-click="currentTab = 0">Input</div>

        <div  ng-class="{`active': currentTab == 1}"
              ng-click="currentTab = 1">Settings</div>
    </div>
    <div ng-switch="currentTab">
      <div ng-switch-when="0"><!-- GESTURES --></div>
      <div ng-switch-when="1"><!-- SETTINGS --></div>
    </div>
  </div>
\end{lstlisting}

\subsubsection{GestureController}

The GestureController detects and interprets the user's swipe gestures. For this purpose it uses Hammer.js\footnote{\url{https://github.com/EightMedia/hammer.js}, last-checked on 22/04/2014}, a gesture-detection library, and the \code{xbmcRemote} service(see \refchapter{subsubsec:xbmcRemote}).

\begin{lstlisting}[language=JavaScript, caption=GestureController mappings, label=lst:gesture-controller]
  hammeredGestureTarget.on('swipeup', function() {
    xbmcRemote.up();
  });
\end{lstlisting}

\reflisting{lst:gesture-controller} \ demonstrates the binding between the \code{hammeredGestureTarget} (a hammer.js target) and the \code{xbmcRemote} service object. Each gesture is associated with one specific action. In this case, a \code{swipeup} would execute the \code{up} command in the XBMC media center, which is used for navigating between elements. Simple swipes e.g. are always used for visual navigation. Another example for gestures is the double tap, it executes the `back' command and navigates to the previous dialog.

\subsubsection{CommandsController}

\subsubsection{SettingsController}

\subsection{Services}

\subsubsection{xbmcRemote}
\label{subsubsec:xbmcRemote}