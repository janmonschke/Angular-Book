%!TEX root = ../../thesis.tex
\section{Directives \& Controllers}

\textbf{Directives} are modules that either create completely new HTML elements with custom layout and functionality, or they can be used to add custom functionality to already existing HTML elements\footnote{\cite[p. 61ff]{lerner2013ngbook}}. If written in a granular way, directives can be reused in many places of an application, so that common functionality does not need to be rewritten. An example for a reused directive in the audio editor is the \code{draggableDirective}, which adds drag and drop functionality to existing HTML elements. The functionality of these elements is described in more detail in \refchapter{impl-recording-piece}. As a rule of thumb, directives should be used whenever there is the need for accessing DOM elements and their events.

\textbf{Controllers} are used to add logic to HTML templates or to directives\footnote{\cite[p. 25ff]{lerner2013ngbook}}. AngularJS comes with its own template engine that supports declarative definitions and declarative two-way bindings. It is best to show the connection of controllers and templates in a short example:

\begin{lstlisting}[language=HTML, caption=A simple AngularJS template, label=lst:angular-template]
  <div ng-controller="MyController">
    <input ng-model="test" />
    <button ng-click="showValue()">Show</button>
  </div>
\end{lstlisting}

The template shown in \reflisting{lst:angular-template} declares that its functionality should be served from a controller called \code{MyController} (line 1). Its HTML input element should be bound to the value of a model called \code{test} (line 2) and a method called \code{showValue} should be called when the button in line 3 is clicked. All the attributes in this example that start with \code{ng-*} are directives that are part of Angular's core\footnote{The example applications in the next chapters will show the most important built-in directives}.

\begin{lstlisting}[language=JavaScript, caption=A simple AngularJS controller, label=lst:angular-controller]
  app.controller("MyController", ["$rootScope", "$scope",
    function($rootScope, $scope){
      $scope.test = "Timah";

      $scope.showValue = function(){
        alert($scope.test);
      };
  }]);
\end{lstlisting}

The context for the referenced model \code{test} in line 2 and the method \code{showValue} is provided by a \textbf{scope} which comes with the controller that is defined in \reflisting{lst:angular-controller}. A scope is used for keeping the application's data and for referencing methods in HTML templates. Furthermore, scopes can also be used to emit and listen to events, similar to DOM events. AngularJS defines which controller to instantiate by looking at the name that is provided from controllers (line 1). The value for the input element from \reflisting{lst:angular-template} comes from the variable that is assigned to the \code{\$scope} in line 3. Its value is bound to the template in two-ways. First, it will always be updated in the template, when \code{\$scope.test} changes in the program's logic. Secondly, whenever the value of the input element is changed by a user, \code{\$scope.test} is updated accordingly. The method that was referenced in the button from the above template is also added to the \code{\$scope} (line 5) and each time the button is pressed, the current value of \code{\$scope.test} will be alerted.