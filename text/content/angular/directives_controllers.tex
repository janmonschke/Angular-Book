%!TEX root = ../../thesis.tex
\section{Directives \& Controllers}
\label{sect:directives_controllers}

\textbf{Directives} are modules that either create completely new HTML elements with custom layout and functionality, or they can be used to add custom functionality to already existing HTML elements\footnote{\cite[p. 61ff]{lerner2013ngbook}}. If written in a granular way, directives can be reused in many places of an application, so that common functionality does not need to be rewritten. An example for a reused directive could be a \code{draggableDirective}, which adds custom drag and drop functionality to existing HTML elements. As a rule of thumb, directives should be used whenever there is the need for accessing DOM elements and their events.

Directives are declared in the \code{app's} namespace, using the \code{directive} method which accepts a name and a function that represents the directive, just as it is shown in \reflisting{lst:angular-directive}.

\begin{lstlisting}[language=JavaScript, caption=A simple AngularJS directive, label=lst:angular-directive]
  app.directive('foobar', function() {
    return {
      restrict: 'A',
      templateUrl: 'foo.html',
      controller: 'BarController',
      link: function(scope, element, attributes){
        scope.displayFoo = function(){
          element[0].innerHTML = "<h1>foo</h1>";
        }
      }
    }
  });
\end{lstlisting}

\begin{lstlisting}[language=HTML, caption=A simple AngularJS directive (HTML), label=lst:angular-directive-html]
  <div ng-click="displayFoo()" foobar>Bar</div>
\end{lstlisting}

The directive function needs to return an object which is then used to enhance the HTML element in the desired way. \reflisting{lst:angular-directive-html} shows an HTML element which has the attribute \code{foobar}. This attribute is used to trigger the creation of the directive which is shown in \reflisting{lst:angular-directive}. Directives can be triggered by attributes (A), element names (E) and by CSS classes (C). The \code{restrict} property (line 3) defines which of these is used as a trigger. A \code{restrict} property of \code{AC} would also trigger the directive from the element name: \code{<foobar></foobar>}. The \code{templateUrl} property (line 4) points to the template that should be inserted into the current element. The \code{controller} property associates a controller to the template if no controller has been specified in the template itself. \code{link} (line 6) defines the function that is used to access the DOM element and to associate it to the scope. In this example, the method \code{displayFoo} is added to the scope and it is executed when the \code{div} element is clicked (line 1, \reflisting{lst:angular-directive-html}). It accessess the element's \code{innerHTML} and adds the headline \code{foo} to it. Directives come in handy when external DOM libraries have to be used in addition to Angular or for DOM events that are not propagated through AngularJS itself. 

\textbf{Controllers} are used to add logic to HTML templates or to directives\footnote{\cite[p. 25ff]{lerner2013ngbook}}. AngularJS comes with its own template engine that supports declarative definitions and declarative two-way bindings. It is best to show the connection of controllers and templates in a short example:

\begin{lstlisting}[language=HTML, caption=A simple AngularJS template, label=lst:angular-template]
  <div ng-controller="MyController">
    <input ng-model="test" />
    <button ng-click="showValue()">Show</button>
  </div>
\end{lstlisting}

The template shown in \reflisting{lst:angular-template} declares that its functionality should be served from a controller called \code{MyController} (line 1). Its HTML input element should be bound to the value of a model called \code{test} (line 2) and a method called \code{showValue} should be called when the button in line 3 is clicked. All the attributes in this example that start with \code{ng-*} are directives that are part of Angular's core\footnote{The example applications in the next chapters will show many important built-in directives}.

\begin{lstlisting}[language=JavaScript, caption=A simple AngularJS controller, label=lst:angular-controller]
  app.controller("MyController", ["$rootScope", "$scope",
    function($rootScope, $scope){
      $scope.test = "Hello";

      $scope.showValue = function(){
        alert($scope.test);
      };
  }]);
\end{lstlisting}

The context for the referenced model \code{test} in line 2 and the method \code{showValue} is provided by a \textbf{scope} which comes with the controller that is defined in \reflisting{lst:angular-controller}. A scope is used for keeping the application's data and for referencing methods in HTML templates. Furthermore, scopes can also be used to emit and listen to events, similar to DOM events. AngularJS defines which controller to instantiate by looking at the name that is provided from controllers (line 1). The value for the input element from \reflisting{lst:angular-template} comes from the variable that is assigned to the \code{\$scope} in line 3. Its value is bound to the template in two-ways. First, it will always be updated in the template, when \code{\$scope.test} changes in the program's logic. Secondly, whenever the value of the input element is changed by a user, \code{\$scope.test} is updated accordingly. The method that was referenced in the button from the above template is also added to the \code{\$scope} (line 5) and each time the button is pressed, the current value of \code{\$scope.test} will be alerted.

In line 3 of \reflisting{lst:angular-template}, the value of the \code{ng-click} directive is an expression and in this case a function call. Expressions are code pieces that will be executed when the directive is evaluated or triggered. In this example, the expression \code{showValue()} is executed when the button is clicked. The context of expressions is always the current scope. In this example, the method \code{showValue} is searched in the current scope and then executed. But expressions can also be single-line assignments like \code{ng-click="test = 'Hello AngularJS'"}, which would set the value of \code{test} in the current scope to \code{Hello AngularJS}. Expressions are a way to descriptively enrich templates with functionality without having to touch controller code or directive code. 