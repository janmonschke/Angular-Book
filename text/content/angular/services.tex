%!TEX root = ../../thesis.tex
\section{Services}
\label{sect:services}

Services are components in AngularJS applications that provide additional logic or bridges to external libraries which are not related to the scope of an element and that should be used in controllers and directives.

\begin{lstlisting}[language=JavaScript, caption=A simple AngularJS service, label=lst:angular-service]
  app.factory('User',
    function($http, $q){
      return {
        bySession: function(){
          // fetch ther user
          // return a promise
          return $q.defer().promise;
        }
      };
  });
\end{lstlisting}

\reflisting{lst:angular-service} shows a \code{User} service that is used to get user data. A service is defined with the \code{factory} method which accepts the name for the service (line 1) and the service function itself. This function returns an object which can then be used in other components and which basically is a singleton in this case. 

The service requires \code{\$http} and \code{\$q} in its parameter declaration which itself are services that are defined by AngularJS. As briefly mentioned before, AngularJS will insert these service using dependency injection. The internal AngularJS module \code{\$injector} takes care of reading parameter declarations and looking up the services. It will then inject the looked up services into the function to make them available in its scope. Dependency injection is a powerful method to drastically simplify testing, because it forces developers to declare a list of dependencies, which helps to understand the code more. And it also makes it really easy to mock dependencies in testing environments.

\begin{lstlisting}[language=JavaScript, caption=Using a custom service, label=lst:angular-service-example]
  app.controller('FooController', function($scope, User){
    $scope.loading = true;
    User.bySession().then(function(userObj){
      $scope.loading = false;
    });
  });
\end{lstlisting}

In \reflisting{lst:angular-service-example}, the \code{User} service (line 1) is injected alongside with the \code{\$scope} service. The example controller (\code{FooController}) is in a \code{loading} state (line 2) initially and then waits for the User service to fetch the current user from the session. It then stops the loading state and the UI can react to it.